\documentclass[12pt,a4paper]{article}
\usepackage[utf8]{inputenc}
\usepackage[T1]{fontenc}
\usepackage{polski}
\usepackage[polish]{babel}
\usepackage{color}
\usepackage{cite}
\usepackage{url}
\usepackage{graphicx}


%element as landscape
\usepackage{pdflscape}
\usepackage{afterpage}

%listings
\usepackage{framed}
\usepackage{listings}
\usepackage[most]{tcolorbox}
\usepackage{filecontents}
\usepackage{tabularx}
\usepackage{caption}
%\tcbuselibrary{listings,breakable}

\lstdefinelanguage{JavaScript}{
  keywords={break, case, catch, continue, debugger, default, delete, do, else, false, finally, for, function, if, in, instanceof, new, null, return, switch, this, throw, true, try, typeof, var, void, while, with},
  morecomment=[l]{//},
  morecomment=[s]{/*}{*/},
  morestring=[b]',
  morestring=[b]",
  ndkeywords={class, export, boolean, throw, implements, import, this},
  keywordstyle=\color{blue}\bfseries,
  ndkeywordstyle=\color{darkgray}\bfseries,
  identifierstyle=\color{black},
  commentstyle=\color{purple}\ttfamily,
  stringstyle=\color{red}\ttfamily,
  sensitive=true
}
\newcounter{listing}
\setcounter{listing}{1}

\newtcbinputlisting{\jsfile}[3][]{%
  listing only,
  %titlepos=b,
  breakable,
  colframe=cyan,
  colback=white,
  left=6mm,
  title=Kod źródłowy \thelisting: #3,
  listing options={
    language=JavaScript,
    basicstyle=\small\ttfamily,
    breaklines=true,
    columns=fullflexible,
    numbers=left,numberstyle=\tiny\color{red!75!black},
    #1
  },
  listing file=#2
}

\newtcblisting{js}[1][]{
  listing only,
  breakable,
  colframe=cyan,
  colback=cyan!10,
  %title=#1,
  listing options={
    language=JavaScript,
    basicstyle=\small\ttfamily,
    breaklines=true,
    columns=fullflexible,
  },
  #1
}

\addto\captionspolish{\renewcommand*{\tablename}{Tabela}}


\begin{document}

\title{Zalecenia odnośnie pisania prac inżynierskich}
\author{Krzysztof Podlaski}
\date{}
\maketitle


\section{Uwagi ogólne}
\begin{itemize}
\item Wszystkie prace muszą spełniać wymagania opisane w regulaminie prac dyplomowych.
\item Część pisemna pracy inżynierskiej może być tworzona w dowolnej technologii (word, latex, openoffice, ...), proszę o przesyłanie do mnie wersji pdf. Osobiście polecam latex ale jest to tylko sugestia :-).
\item W trakcie pracy nad treścią pracy nie poprawiam tekstu, jedynie dodaję komentarze do dokumentu pdf.
\item Proszę o założenie na potrzeby projektu repozytorium git/bitbucket, abym miał dostęp do aktualnej wersji kodu programistycznego.  Repozytorium może być publiczne lub prywatne, w przypadku prywatnego proszę o dodanie mnie do tego repozytorium.
\item W trakcie pracy nad projektem inżynierskim proszę pamiętać o tym co nauczyliście się w trakcie przedmiotu inżynieria oprogramowania. Nie wymagam pracy w modelu kaskadowym (waterfall). Pewnie wygodniejsza jest jedna z metodyk iteracyjnych, zwinnych. Jednakże, każda z~nich wymaga aby każda z poniższych faz występowała co najmniej jednokrotnie:
    \begin{itemize}
    \item analizy wymagań,
    \item projektowania,
    \item implementacji,
    \item testowania.
    \end{itemize}
    W przypadku stosowania modelu iteracyjnego często fazy występują kilkukrotnie. Wymienione fazy muszą/powinny mieć swoje odzwierciedlenie w części pisemnej pracy inżynierskiej. Z mojego punktu widzenia bardzo ważne są dwie pierwsze oraz ostatnia z faz, one odzwierciedlają profesjonalizm dyplomanta. Jakość pracy wykonanej w trakcie implementacji świadczy o kompetencjach technologicznych.
\item Programistyczna praca inżynierska wymaga dobrej dokumentacji projektu (patrz Sekcja \ref{dokumentacja}).
\item Zalecam tworzenie bibliografii w miarę pisania pracy. Podejście ``dodam później'', niestety stosowane, zwykle wychodzi bokiem.
\item To samo dotyczy formatowania, w przypadku korzystania z word bądź openoffice zalecam od samego początku wykorzystywać odpowiednie style, aby można było później łatwo zmieniać format całego dokumentu. Zostawianie tego na sam koniec powoduje często problemy i ``rozjeżdżanie się'' pracy.
\end{itemize}

\section{Forma}

\subsection{Język}
Praca powinna być napisana poprawnie w języku polskim. Proszę zwrócić uwagę na wszelakie anglicyzmy często występujące w slangu informatycznym. W miarę możliwości stosujmy polskie słowa, jeżeli nie da rady łatwo zastąpić słowa angielskiego, należy tak przeformułować zdanie aby nie była wymagana odmiana słowa, a co za tym idzie konieczności dodawania polskiej końcówki do słowa angielskiego.
\begin{description}
  \item[{\color{red}Źle:}] Na potrzeby pracy usługi \underline{RESTowe} zostały stworzone z wykorzystaniem \underline{frameworka} Spring.
  \item[Poprawnie:] Na potrzeby pracy w celu stworzenia usług typu \underline{REST} wykorzystano \underline{framework} Spring.
\end{description}

Dodatkowo sugeruję stosowanie twardych spacji w celu unikania wszelkiego rodzaju sierotek, przed ostatecznym wysłaniem pracy do APD proszę o sprawdzenie czy nie występują  bękarty, sierotki, wdowy i szewcy \cite{Bekart}.

\subsubsection*{Poprawność językowa}
Dodatkowo proszę o zwrócenie szczególnej uwagi na stronę edycyjną pracy, ograniczenie do minimum błędów gramatycznych bądź ortograficznych. Oczywiście błędy są niedopuszczalne jednak się zdarzają. Zalecam więc wykorzystanie automatycznego sprawdzenia poprawności językowej. Poziom weryfikacji pisowni mocno zależy od środowiska \LaTeX. Na przykład można wykorzystać słowniki z środowiska OpenOffice (pl\_PL.aff oraz pl\_PL.dic) skopiować je  do katalogu {\tt MiKTeX/hunspell/dicts}, wtedy w wybranym edytorze można włączyć możliwość podkreślania błędów po ustawieniu polskiego słownika.

\subsection{Rozdziały}
Nie ma ogólnych założeń co do podziału logicznego pracy, przykładowy podział pracy na rozdziały poświęcone kolejnym zagadnieniom:
\begin{enumerate}
\item Wstęp - powinien zawierać ogólny opis celów pracy
\item Analiza wymagań,
\item Opis metod i technologii wykorzystanych w pracy,
\item Opis implementacji (dokumentacja projektu),
\item Testy aplikacji,
\item Instrukcja instalacji/obsługi,
\item Podsumowanie,
\item Bibliografia.
\end{enumerate}

\subsection{Odwołania}

\subsubsection{Bibliograficzne}
Najpopularniejsze są trzy systemy cytowań (Vancouver System, Harvard System, Oxford System) opisane szczegółowo w \cite{Cytowanie}, {\bf regulamin dyplomowania WFiIS nakazuje jednoznacznie  wykorzystanie w pracy dyplomowej systemu \underline{Vancouver}}. W~przypadku systemów Vancouver i Oxford bibliografia powinna być posortowana w kolejności pojawiania się odwołań w tekście, natomiast w systemie Harvard alfabetycznie wedle nazwisk pierwszych autorów.


\subsubsection{Elementy pływające}
Każdy element pływający jak: obrazek, schemat, tabela czy fragment z~kodem źródłowym powinien zostać odpowiednio oznaczony (ponumerowany) i~zatytułowany. W~pracy powinno znaleźć się co najmniej jedno odwołanie do wspomnianego elementu, w~innym wypadku sprawia to wrażenie, iż element jest zbędny. Dodatkowo należy pamiętać, iż zasady umieszczania elementów pływających w~tekście ograniczają znacząco jaką część strony mogą zawierać, a~co za tym idzie omawiana tabelka, obrazek nie muszą znajdować się w~bezpośrednim sąsiedztwie tekstu odnoszącego się do obiektu.

\section{Dokumentacja Projektu}\label{dokumentacja}

\subsection{Wymagania aplikacji}
Praca powinna zawierać analizę wymagań aplikacji, każda forma stosowana w~inżynierii oprogramowania jest odpowiednia \cite{IO}, dodatkowo do opisu założeń aplikacji należało by wykorzystać przypadki użycia (Use Cases) \cite{UML}. Do tworzenia diagramów można wykorzystać dowolne narzędzia, wiele środowisk posiada odpowiednie narzędzia. Praktycznie każdy rodzaj diagramu UML można stworzyć za pomocą  prostego środowiska PlantUML \cite{plantUML}.

\begin{figure}[h]
\centering
\includegraphics[width = \textwidth]{DiagramKlasPakiet.jpg}
\caption{Przykładowy diagram klas w wybranym pakiecie i zależności pomiędzy nimi.}\label{diagram_klas_pakiet}
\end{figure}


\subsection{Architektura aplikacji/systemu}
Bardzo często aplikacja/system tworzony w ramach pracy inżynierskiej składa się z wielu elementów jak np. baza danych, aplikacja serwerowa, aplikacja kliencka (mobilna). W pracy należy wyraźnie przedstawić podział na elementy składowe i metodę komunikacji pomiędzy nimi. Zalecanym sposobem przedstawienia zależności pomiędzy częściami projektu, oprócz opisu słownego, jest stosowanie schematów.

\subsection{Dokumentacja kodu}
 Poprawnym sposobem dokumentacji struktury klas jest diagram klas. Możemy stosować różnego rodzaju diagramy, bardzo ogólny reprezentujący strukturę całej aplikacji Rys.~\ref{duzy_diagram_klas}, jak i diagramy opisujące poszczególne pakiety Rys.~\ref{diagram_klas_pakiet} bądź pojedyncze klasy Rys.~\ref{diagram_wybranej_klasy}. Do diagramów klas należy dołączyć bardziej szczegółowy opis klasy, informujący użytkownika za co klasa odpowiada, jakie są jej pola, metody i ich parametry, w tym celu można wykorzystać postać tabelaryczną (Tab.~\ref{Opis_Klasy}) lub przedstawić opis w tekście pracy. W~przypadku prac zawierających znaczną ilość klas i ich metod nie wymagam aby wszystkie metody/pola zostały opisane w pracy. W~części  pisemnej należy opisać jedynie najistotniejsze metody/pola, dodatkowo należy stworzyć dokumentację techniczną klas za pomocą odpowiedniego narzędzia (doxygen, javadoc, ...) i~dodać do pracy w postaci załącznika w formie elektronicznej (na płycie CD/DVD).


\begin{table}[h!b]
\begin{center}
\begin{tabularx}{\textwidth}{|l|X|}
  \hline
  % after \\: \hline or \cline{col1-col2} \cline{col3-col4} ...
  Nazwa metody  & paymentOut  \\\hline
  Opis metody   & dokonanie przez użytkownika wpłaty na konto bankowe\\\hline\hline
  \multicolumn{2}{|c|}{parametry wejściowe} \\\hline
  user: User            &  reprezentuje użytkownika dokonującego wpłaty  \\\hline
  ammount: double       &  kwota do wpłacenia  \\\hline
  description : String  & opis wpłaty \\\hline
  accountId: int        & numer id konta na które należy dokonać wpłaty środków \\\hline\hline
  \multicolumn{2}{|c|}{parametry wyjściowe}   \\\hline
  typ boolean &  czy operacja zakończyła się sukcesem\\\hline\hline
  \multicolumn{2}{|c|}{Uwagi dodatkowe}\\\hline
  możliwe wyjątki   & \textit{OperationIsNotAllowedException} \\
                    & wyjątek tworzony w przypadku gdy użytkownik nie ma prawa wypłaty z podanego konta\\
                    & \textit{SQLException}\\
                    & wyjątek tworzony w przypadku kiedy nie ma  konta o zadanym accountid\\\hline
  podejmowane działania & W trakcie działania metody wszelkie operacje  podlegają logowaniu z wykorzystaniem pola history\\
  \hline
\end{tabularx}
\caption{Szczegółowy opis metody paymentOut klasy AccountManager.}\label{Opis_Klasy}
\end{center}
\end{table}

\begin{figure}[!h]
\centering
\includegraphics[width = 200pt]{DiagramKlasy.jpg}
\caption{Przykładowy diagram opisujący jedną klasę.}\label{diagram_wybranej_klasy}
\end{figure}


Działanie aplikacji, kodu aplikacji, powinno zostać zaprezentowane z wykorzystaniem diagramów UML\cite{UML}: sekwencji (Sequence diagram), aktywności (Activity diagram), stanu (State diagram) bądź schematów blokowych.  Jeżeli autor pracy uzna, że fragment kodu jest niezbędny, należy zastosować krótki jego fragment (do 10-20 linijek kodu). Proszę o unikanie wklejania do dokumentu fragmentów kodu źródłowego aplikacji. Dodatkowo nie zalecam używania w tym celu zrzutów ekranu z środowiska programistycznego, powoduje to często różne rozmiary czcionek w zależności od rozmiaru wyciętego fragmentu. Dodatkowo kod powinien być kolorowany na białym tle, nie należy stosować motywów typu kolorowy tekst na czarnym tle - wygląda to dużo gorzej na papierze i zużywa mnóstwo tonera, tuszu. Mile widziana jest numeracja linii co ułatwia późniejsze odnoszenie się do przedstawionego kodu źródłowego. Kolorystyka i obramowanie kodu nie musi przypominać tego umieszczonego w dokumencie (Kod~źr.~1), jednakże kod źródłowy musi się wyraźnie odróżniać od reszty tekstu pracy.

\jsfile[label=kod1]{code3.js}{Kod źródłowy obiektu app}



\subsection{Dokumentacja baz danych}
Praca musi zawierać informację o strukturze wykorzystywanych baz danych. W przypadku stosowania baz relacyjnych zaleca się postać diagramów opisujących tabele i relacje pomiędzy nimi, w natomiast przypadku baz obiektowych diagramy obiektów.

\afterpage{%
    \clearpage% Flush earlier floats (otherwise order might not be correct)
    \thispagestyle{empty}% empty page style (?)
    \begin{landscape}% Landscape page
        \begin{figure}
            \centering % Center table
            \includegraphics[width=1.5\textwidth]{DiagramKlasBig.jpg}
            \caption{Przykładowy ogólny diagram klas dla całej aplikacji.}\label{duzy_diagram_klas}
        \end{figure}
    \end{landscape}
    \clearpage% Flush page
}



\subsection{Protokoły wymiany informacji}
Jeżeli aplikacja zakłada wymianę danych pomiędzy elementami systemu, lub aplikacjami zewnętrznymi np z wykorzystaniem SOAP lub REST, należy opisać dokładnie protokół wymiany informacji, dla przykładu opis protokołu REST zawierają Tabele~\ref{rest1},~\ref{rest2}.

\begin{table}[th]
\begin{tabularx}{\textwidth}{|l|X|}
\hline
&\textbf{Pobranie wybranego rekordu}\\\hline
URL &   \url{http://geniusgamedev.eu/cordova/rest_api/rest_srv/:id} lub \url{http://geniusgamedev.eu/cordova/rest_api/rest.php?id=:id}\\\hline
Method  & GET\\\hline
Parameters  & id - id wybranego elementu \\\hline
Request Data & None\\\hline
Response Data & obiekt w postaci:\\
&

%\begin{js}
\{'data':\{
    'id':2,
    'name':'Helena',
    'score':128
    \}
\}
%\end{js}

\\\hline
\end{tabularx}
\caption{Metoda get stosowanej usługi REST}\label{rest1}
\end{table}

\begin{table}
\begin{tabularx}{\textwidth}{|l|X|}
\hline
&\textbf{Dodanie nowego rekordu}\\\hline
URL &   \url{http://geniusgamedev.eu/cordova/rest_api/rest_srv} or \url{http://geniusgamedev.eu/cordova/rest_api/rest.php}\\\hline
Method  & POST\\\hline
Parameters  & brak \\\hline
Request Data & nowy obiekt w postaci:\\
&\{
'data': \{
'id': 1,
'name': 'Jane',
'score': 332
\}
\}
\\\hline
Response Data & obiekt w postaci:\\
&
\{'data':\{
    'id':12,
    'name':'Jane',
    'score':332
    \}
\}
\\
&nowa wartość id została przypisana przez serwer w trakcie rejestracji obiektu.
\\\hline
\end{tabularx}
\caption{Metoda post stosowanej usługi REST}\label{rest2}
\end{table}



\subsection{Dokumentacja interfejsu użytkownika}
Dokumentacja interfejsu użytkownika powinna pokazywać przepływ pomiędzy widokami, na przykład za pomocą diagramu stanów (State diagram), dodatkowo możliwe jest wykorzystanie schematów typu UI-mockups reprezentujących rozłożenie elementów widoku. Obrazy przedstawiające rzeczywisty wygląd ekranów aplikacji (zrzuty ekranów) powinny raczej zostać umieszczone w instrukcji obsługi.


\subsection{Dokumentacja Testów}

Na potrzeby pracy należy wykonać testy manualne i/lub automatyczne. Mile widziane jest zastosowanie testów jednostkowych, interfejsu użytkownika czy akceptacyjnych. Testy należy opisać w postaci: wyszczególnionych przypadków testowych, każdy przypadek testowy powinien zawierać opis sposobu wykonania, kroków niezbędnych do jego przeprowadzenia.


\section{Egzamin dyplomowy -- obrona}

Komisja egzaminacyjna składa się z trzech osób, przewodniczącego, recenzenta oraz opiekuna pracy. Egzamin składa się z dwu elementów prezentacji efektów pracy dyplomowej oraz faktycznego egzaminu.

\subsection{Prezentacja efektów pracy dyplomowej}
Dyplomant/dyplomantka jest zwykle proszony o prezentacje efektów pracy dyplomowej, w przypadku pracy inżynierskiej jest to zwykle aplikacja/system w przypadku prac magisterskich są to często efekty części badawczej. Student/studentka ma za zadanie w trakcie 5-10 minut zaprezentować najważniejsze elementy pracy.

\begin{itemize}
\item Część prezentacyjna może być wsparta prezentacją (ppt, pdf), zalecam taką formę wspomaga ona studenta/studentkę i zapobiega zgubieniu się pod wpływem stresu. W ramach potrzeb można przygotować krótki film wizualizujący działanie aplikacji, prezentacje aplikacji na żywo zwiększają ryzyko problemów technicznych.

\item Prezentacja powinna się skupić na najważniejszych efektach pracy. Proszę więc o unikanie prezentacji wszystkich szczegółów/ekranów aplikacji, lepiej skupić się na spisie funkcjonalności, można zaprezentować jeden/kilka ekranów w celu pokazania przyjętej formy interfejsu użytkownika.

\item Przedstawiana praca dyplomowa jest z zakresu informatyki proszę więc skupić się na elementach informatycznych pracy -- ale nie chodzi tu o elementy techniczne a fundamentalne jak stosowane algorytmy, struktury danych/klas natomiast nie jaka instrukcja pomogła w wykonaniu np. sortowania.

\item W trakcie prezentacji proszę unikać sformułowań czysto slangowych bez wyjaśnienia ``JSON'', ``AJAX'', ``Spring Beans''. Pomimo iż jesteśmy informatykami proszę próbować prezentować z założeniem średniej specjalizacji słuchaczy w dziedzinie pracy dyplomowej.
\end{itemize}

\subsection{Część egzaminacyjna}

Student/studentka dostaje trzy pytania, po jednym od każdego z członków komisji egzaminacyjnej. Przewidywany jest czas na przygotowanie do odpowiedzi, nie należy jednak z tym przesadzać normą jest max 5-7 minut przygotowania. Po zakończeniu przygotowania dyplomant/dyplomantka odpowiada na pytania w dowolnej kolejności.

W trakcie, jak i po zakończeniu, odpowiedzi na pytanie członkowie komisji mogą zadawać dodatkowe pytania uszczegóławiające lub wspomagające. Student przechodzi do następnego pytania po jawnej informacji, że komisja nie ma nic do dodania do aktualnego pytania.






\thebibliography{99}



\bibitem{Bekart} Strony Wikipedii poświęcone błędom w łamaniu tekstu\\
\url{https://pl.wikipedia.org/wiki/B%C4%99kart_(typografia)}

\bibitem{Cytowanie} Strona Wikipedii poświęcona metodom cytowania \\
\url{https://pl.wikipedia.org/wiki/Cytowanie_pi%C5%9Bmiennictwa}

\bibitem{IO} Ian Sommerville, Software Engineering, 10h ed., Pearsons  (2016)\\
 wydanie polskie Ian Sommerville, Inżynieria Oprogramowania, WNT (2003)

\bibitem{UML} Pascal Roques, UML in Practice: The Art of Modeling Software Systems Demonstrated through Worked Examples and Solutions, John Wiley \& Sons, 2004\footnote{
    Przykładowy rozdział książki dotyczący wybranych diagramów UML i przypadków użycia \url{https://people.ok.ubc.ca/bowenhui/310/8-UML.pdf}}

\bibitem{plantUML} Strona domowa projektu PlantUML \url{https://plantuml.com}



\end{document}
